%%%%%%%%%%%%
%
% $Autor: Wings $
% $Datum: 2019-03-05 08:03:15Z $
% $Pfad: MainFunction.tex $
% $Version: 4250 $
% !TeX spellcheck = en_GB/de_DE
% !TeX encoding = utf8
% !TeX root = manual 
% !TeX TXS-program:bibliography = txs:///biber
%
%%%%%%%%%%%%

\chapter{\TRANS{Hauptfunktion}{Main Function}}

In this section, we describe the key functionalities and workflow of the Hurricane Intensity Prediction System from both the developer and user perspectives.

The main functions of the system are summarized in Table~\ref{tab:main_functions}.

\begin{table}[h]
	\centering
	\begin{tabular}{|c|p{8cm}|}
		\hline
		\textbf{Functionality} & \textbf{Description} \\
		\hline
		Data Preprocessing & Load and clean the NOAA Atlantic Hurricane Dataset or user-provided data to ensure compatibility with the models. \\
		\hline
		Model Selection & Developer selects and configures ARIMA and LSTM models for hurricane intensity prediction. \\
		\hline
		Training (Developer) & Models are trained on historical hurricane data using optimal parameters set by the developer. Trained models are saved in the \texttt{models/} directory. \\
		\hline
		Prediction (User) & User uploads new hurricane data via the Streamlit app. The system loads the pre-trained models and generates predictions automatically. \\
		\hline
		Visualization & The application automatically visualizes prediction results for the user within the web interface. \\
		\hline
		Evaluation (Developer) & Developer can compare predicted values with actual values to assess and improve model performance. \\
		\hline
	\end{tabular}
	\caption{Main Functions of the Hurricane Intensity Prediction System}
	\label{tab:main_functions}
\end{table}

\section*{Workflow Overview}

The following diagram illustrates the workflow for both developers and users:

\begin{center}
	\begin{tikzpicture}[node distance=2.5cm, every node/.style={font=\small}]
		% Developer side
		\node (dev) [draw, rectangle, fill=blue!10, minimum width=3cm] {Developer Workflow};
		\node (preprocess) [draw, rectangle, below of=dev, fill=gray!10] {Data Preprocessing};
		\node (select) [draw, rectangle, below of=preprocess, fill=gray!10] {Model Selection \& Parameter Setting};
		\node (train) [draw, rectangle, below of=select, fill=gray!10] {Model Training};
		\node (save) [draw, rectangle, below of=train, fill=gray!10] {Save Trained Models (\texttt{models/})};
		
		% User side
		\node (user) [draw, rectangle, right=5cm of dev, fill=green!10, minimum width=3cm] {User Workflow};
		\node (upload) [draw, rectangle, below of=user, fill=gray!10] {Upload Data (Streamlit App)};
		\node (predict) [draw, rectangle, below of=upload, fill=gray!10] {Load Trained Models \& Predict};
		\node (visualize) [draw, rectangle, below of=predict, fill=gray!10] {Automatic Visualization};
		\node (download) [draw, rectangle, below of=visualize, fill=gray!10] {Download Predictions};
		
		% Arrows for developer
		\draw[->, thick] (dev) -- (preprocess);
		\draw[->, thick] (preprocess) -- (select);
		\draw[->, thick] (select) -- (train);
		\draw[->, thick] (train) -- (save);
		
		% Arrows for user
		\draw[->, thick] (user) -- (upload);
		\draw[->, thick] (upload) -- (predict);
		\draw[->, thick] (predict) -- (visualize);
		\draw[->, thick] (visualize) -- (download);
		
		% Connection between developer and user
		\draw[->, thick, dashed] (save.east) -- ++(1,0) -- ++(0,5.5) -- (predict.west);
		
		% Labels
		\node[above=0.2cm of dev] {\textbf{Developer}};
		\node[above=0.2cm of user] {\textbf{User}};
	\end{tikzpicture}
\end{center}

\noindent
\textbf{Legend:} \\
\textbf{Developer:} Responsible for data preprocessing, model selection, training, and saving models. \\
\textbf{User:} Interacts with the application via Streamlit, uploads data, and receives predictions and visualizations using pre-trained models.




