%%%%%%
%
% $Autor: Wings $
% $Datum: 2020-01-18 11:15:45Z $
% $Pfad: WuSt/Skript/Produktspezifikation/powerpoint/ImageProcessing.tex $
% $Version: 4620 $
%
%%%%%%

\definecolor{MapleColor}{rgb}{1,0.0,0.}
\definecolor{PythonColor}{rgb}{0,0.5,1.}
\definecolor{ShellColor}{rgb}{1,0,0.5}
\definecolor{FileColor}{rgb}{0.5,0.5,1.}

\newcommand{\MapleCommand}[1]{\textcolor{MapleColor}{\texttt{#1}}}
\newcommand{\PYTHON}[1]{\textcolor{PythonColor}{\texttt{\justify#1}}}
\newcommand{\SHELL}[1]{#1}%\textcolor{ShellColor}{\texttt{\justify#1}}}
\newcommand{\FILE}[1]{\textcolor{FileColor}{\texttt{\justify#1}}}


\newcommand{\QUELLE}{\textcolor{red}{hier Quelle finden}}

\newcolumntype{L}[1]{>{\raggedright\arraybackslash}p{#1}} % linksbündig mit Breitenangabe

\newcommand{\GRAFIK}{\textcolor{red}{Grafik einfügen}}

\newcommand{\DEF}[1]{\fcolorbox{blue}{blue!10}{\begin{minipage}{\textwidth}\textbf{Definition.}#1\end{minipage}}}

\newcommand{\BEISPIEL}[1]{\fcolorbox{blue}{blue!10}{\begin{minipage}{\textwidth}\textbf{Beispiel.}#1\end{minipage}}}

\newcommand{\SATZ}[1]{\fcolorbox{blue}{blue!10}{\begin{minipage}{\textwidth}\textbf{Satz.}#1\end{minipage}}}

\newcommand{\Bemerkung}[1]{\fcolorbox{blue}{blue!10}{\begin{minipage}{\textwidth}\textbf{Bemerkung.}#1\end{minipage}}}


\newcommand{\Po}{\mathbb{P}}
\newcommand{\R}{\mathbb{R}}

\DeclareMathOperator{\Atan2}{Atan2}
\DeclareMathOperator{\sign}{sign}

% Auswahl der Sprache
% 1.Argument ist der Pfad ohne "en" oder "de"
% 2.Argument ist der Dateiname
\newcommand{\InputLanguage}[2]{
    \ifdefined\isGerman
    \input{#1de/#2}
    \else
    \ifdefined\isEnglish
    \input{#1en/#2}
    \else
    \input{#1de/#2}
    \fi
    \fi
}

\newcommand{\TRANS}[2]{
    \ifdefined\isGerman
    #1%
    \else
    \ifdefined\isEnglish
    #2%
    \else
    #1%
    \fi
    \fi
}


% muss für Akronyme \ac statt see verwendet werden.
\newcommand{\Siehe}{
    \ifdefined\isGerman
    \emph{siehe}
    \else
    \ifdefined\isEnglish
    \emph{see}
    \else
    \emph{siehe}
    \fi
    \fi
}

%todo Die Kommandos sind für das Endprodukt zu entfernen. Die entsprechenden Stellen sind zu bearbeiten bzw. zu löschen
\newcommand{\Mynote}[1]{\marginnote{\textcolor{red}{WS:#1}}}
\newcommand{\Ausblenden}[1]{}
\newcommand{\ToDo}[1]{\textcolor{red}{\section{ToDo} #1}}
% Fehlerzähler
\setul{0.5ex}{0.3ex}
\setulcolor{red}
\newcounter{fehlernummer}
\setcounter{fehlernummer}{11}
\newcommand{\FEHLER}[1]{\ul{#1}\stepcounter{fehlernummer}\textsuperscript{\textcolor{red}{\arabic{fehlernummer}}}}
%\renewcommand{\FEHLER}[1]{\ul{#1}\stepcounter{fehlernummer}\marginnote{\textcolor{red}{\arabic{fehlernummer}}}} geht leider nicht
%\renewcommand{\FEHLER}[1]{\ul{#1}\stepcounter{fehlernummer}\footnote{\textcolor{red}{\arabic{fehlernummer}.Fehler}}} geht leider nicht


\lstdefinestyle{all}{
	alsoletter={-},
	frame=single, 	% top,frame=bottom,
	numbers=none,
	numberstyle=\tiny\color{textc},
	basicstyle=\linespread{0.9}\ttfamily\footnotesize\color{textc},
	tabsize=4,
	showstringspaces=false,
	captionpos=t,
	rulecolor=\color{lightgray!40},
	keywordstyle=\color{keywordc},
	stringstyle=\color{stringc},
	commentstyle=\color{commentc},
	breaklines=true,
	escapechar="!",
	postbreak=\mbox{\textcolor{green}{$\hookrightarrow$}\space},
}
\lstset{
	language=Python,
	basicstyle=\ttfamily,
	keywordstyle=\bfseries,
	commentstyle=\itshape,
	numbers=left,
	numberstyle=\tiny,
	numbersep=5pt,
	breaklines=true,
	breakatwhitespace=true,
	tabsize=2,
	frame=single,
}

\lstdefinestyle{bashstyle}{
	style=all,
	keywords=[2]{-y, --no-install-recommends, --allow-change-held-packages, --allow-downgrades, --fetch-keys, -n, --version, --params, -c, -i, -O, --upgrade, --no-cache-dir, --extra-index-url, --show, -s, -m},
	keywordstyle=[2]\color{parameterc},
	morekeywords = {ln,choco,pip,pip3,apt,apt-key,apt-get,apt-mark,add-apt-repository,wget,mktemp,dpkg,dpkg-query,echo,>>,rm,tegrastats, systemctl},
	deletekeywords={local,LOCAL},
}

\lstdefinestyle{pythonstyle}{
	style=all,
	morekeywords={as},
	keywords=[2]{True, False, None},
	keywordstyle=[2]\color{typec},
	alsoletter={_},
	keywords=[3]{max_workspace_size_bytes, precision_mode, maximum_cached_engines, use_calibration, optimizer, loss, input_shape, from_logits, metrics, batch_size, epochs, validation_data, activation, use_calibration, filters, kernel_size, pool_size, units},
	keywordstyle=[3]\color{parameterc},
	deletekeywords={compile,COMPILE},
}

\lstdefinestyle{inlinestyle}{
	style=all,
	breaklines        = true,
	breakatwhitespace = true,
	breakindent       = 2ex,
	escapechar        = *,
	numbers           = left,
	postbreak=,
}
\lstdefinelanguage{MyBash} {
	language = Bash,
	style=bashstyle,
}

\lstdefinelanguage{MyPython} {
	language = Python,
	style=pythonstyle,
}


\definecolor{PythonColor}{rgb}{0,0.5,1.}
\newcommand{\PYTHONHL}[1]{\textcolor{PythonColorHighLite}{\texttt{#1}}}
\definecolor{MapleColor}{rgb}{1,0,0}

