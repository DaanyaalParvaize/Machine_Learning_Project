
%%%%%%
%
% $Autor: Wings $
% $Datum: 2020-01-18 11:15:45Z $
% $Pfad: WuSt/Skript/Produktspezifikation/powerpoint/ImageProcessing.tex $
% $Version: 4620 $
%
%%%%%%

\chapter{AppDev-Data}


\section{Application Data}
This section describes the data utilized by the hurricane intensity prediction application, as implemented in \texttt{app.py}, \texttt{developer.py}, \texttt{model\_utils.py}, and \texttt{train\_models.ipynb}, focusing on its structure, size, format, anomalies, and origin.

\subsection{Structure}
The data is a univariate time-series comprising wind speed measurements and associated timestamps, organized as a tabular dataset with two columns: \texttt{date} (timestamp) and \texttt{wind\_speed} (numeric value in mph), processed in \texttt{model\_utils.py: load\_storm\_data}.

\subsection{Size}
The dataset typically contains 10 to 1000 observations, with a minimum of 10 required for ARIMA and 11 (sequence length + 1) for LSTM, as validated in \texttt{app.py}. File sizes range from a few KB to several MB, depending on the number of records.

\subsection{Format}
Data is stored in CSV format, with \texttt{date} in ISO 8601 (e.g., \texttt{2023-01-01}) or parseable formats (e.g., \texttt{YYYY-MM-DD}) and \texttt{wind\_speed} as floating-point or integer values, loaded via \texttt{pandas} in \texttt{model\_utils.py}.

\subsection{Anomalies}
Potential anomalies include missing values, non-numeric entries, or duplicate timestamps, addressed in \texttt{model\_utils.py} through forward/backward filling for missing data, coercion of non-numeric values to NaN, and chronological sorting to resolve duplicates.

\subsection{Origin}
The data originates from user-uploaded CSV files containing historical hurricane wind speed records, input via \texttt{app.py}'s interface, with preprocessing and validation handled in \texttt{model\_utils.py} and \texttt{train\_models.ipynb}.